\documentclass{article}
\usepackage[utf8]{inputenc}
\usepackage[sc]{mathpazo}

\usepackage[T1]{fontenc}
% \renewcommand{\familydefault}{\sfdefault}

\usepackage[pdftex,
            pdfauthor={LE Dinh-Viet-Toan},
            pdftitle={Title},
            pdfcreator={pdflatex},
            hidelinks]{hyperref}

\usepackage{ragged2e}
\usepackage[a4paper, right=2.5cm,left=2.5cm,top=2.5cm,bottom=2.5cm]{geometry}

\usepackage{mathtools} % Basic maths
\usepackage{mathabx}
\usepackage{indentfirst} % Indentation at the beginning of a section
\usepackage{textcomp} % Extra characters
\usepackage{todonotes} 
\usepackage{multicol} % Multi columns
\usepackage{cancel} % Draw line to cancel a term

\usepackage[export]{adjustbox}
\usepackage{subcaption}

% Graphics
\usepackage{tikz}
\usetikzlibrary{shapes,arrows,shadings,patterns}
\usepackage{natbib}
\usepackage{graphicx,wrapfig,lipsum}

% Disable number in section
\let\oldsection\section
\renewcommand{\section}[1]{\oldsection*{#1}}
\let\oldsubsection\subsection
\renewcommand{\subsection}[1]{\oldsubsection*{\hspace{1em}#1}}

% Extra commands
\newcommand\norm[1]{\left\lVert#1\right\rVert}
\newcommand{\pplus}{p^{+}}
\newcommand{\pmoins}{p^{-}}
\newcommand{\sqrttwoovtwo}{\frac{\sqrt{2}}{2}}

% Math operators
\DeclareMathOperator\Imspace{Im}

\providecommand{\dividespace}{\vspace{1.6em}}

\title{Démonstration rotation de 45° entre $(\pplus,\pmoins)$ et $(p,u)$ }
\author{}
\date{}



\begin{document}

\maketitle

\section{Équation des ondes 1D}

Soit l'équation des ondes 1D : 
\begin{equation*}
	\frac{\partial^2 P}{\partial t^2} - c^2 \frac{\partial^2 P}{\partial x^2} = 0
\end{equation*}

\subsection{Solution en pression}
Soit $P(x,t) = f(x-ct) + g(x+ct)$ où $f$ et $g$ fonctions quelconques.

On a que 
\begin{align*}
	\frac{\partial^2 f(x-ct)}{\partial t^2} - c^2 \frac{\partial^2 f(x-ct)}{\partial x^2} = c^2 f''(x-ct) - c^2 f''(x-ct) = 0
\end{align*}

Et

\begin{align*}
	\frac{\partial^2 g(x+ct)}{\partial t^2} - c^2 \frac{\partial^2 g(x+ct)}{\partial x^2} = c^2 g''(x+ct) - c^2 g''(x+ct) = 0
\end{align*}

On a bien, par somme que $P(x,t) = f(x-ct) + g(x+ct)$ est solution de l'équation des ondes 1D.

On posera $\pplus = f(x-ct)$ et $\pmoins = g(x+ct)$.

\subsection{Solution en débit}
Par Euler linéarisé :

\begin{align*}
	\rho_0 \frac{\partial \vec{v}}{\partial t} 
	& = -\vec{\nabla}{P} \\
	& = - \left[\frac{\partial \pplus}{\partial x} + \frac{\partial \pmoins}{\partial x}\right].\vec{u_x} \\
	& = - \left[\frac{\partial f(x-ct)}{\partial (x-ct)} \frac{\partial (x-ct)}{\partial x} + 
				\frac{\partial g(x+ct)}{\partial (x+ct)} \frac{\partial (x+ct)}{\partial x} \right].\vec{u_x} \\
	& = - \left[\frac{\partial f(x-ct)}{\partial (x-ct)} + \frac{\partial g(x+ct)}{\partial (x+ct)} \right].\vec{u_x}
\end{align*}

Puis en intégrant et en projetant sur $\vec{u_x}$ :

\begin{align*}
	\rho_0 v 
	& = - \int \frac{\partial f(x-ct)}{\partial (x-ct)} + \frac{\partial g(x+ct)}{\partial (x+ct)} dt \\
	& = -\left[-\frac{1}{c} f(x-ct) + \frac{1}{c} g(x+ct)\right] \\
	& = \frac{1}{c} f(x-ct) - \frac{1}{c} g(x+ct) 
\end{align*}

D'où :
\begin{equation*}
	v = \frac{1}{\rho_0 c} (\pplus - \pmoins)
\end{equation*}

Et en utilisant le débit $u=v.S$ et l'impédance acoustique caractéristique $Z_c = \frac{\rho_0 c}{S}$ :
\begin{equation*}
	u = Z_c^{-1} (\pplus - \pmoins)
\end{equation*}

\section{Lien $(\pplus,\pmoins)$ et $(p,u)$}

On a donc que :

\begin{equation*}
	\begin{cases}
		P & = \pplus + \pmoins \\
		Z_c.u & = \pplus - \pmoins
	\end{cases}
\end{equation*}

D'où : 

\begin{align*}
	\begin{pmatrix}
		P\\
		Z_{c} .u
	\end{pmatrix} \ 
	& =\ 
	\begin{pmatrix}
		1 & -1\\
		1 & 1
	\end{pmatrix}
	\begin{pmatrix}
		\pplus\\
		-\pmoins
	\end{pmatrix} \\
	& =\
	\frac{2}{\sqrt{2}}
	\begin{pmatrix}
		\sqrttwoovtwo & -\sqrttwoovtwo\\
		\sqrttwoovtwo & \sqrttwoovtwo
	\end{pmatrix}
	\begin{pmatrix}
		\pplus\\
		-\pmoins
	\end{pmatrix} \\
	\begin{pmatrix}
		P\\
		Z_{c} .u
	\end{pmatrix}
	& =\
	\frac{2}{\sqrt{2}} R_{\frac{\pi}{4}}
	\begin{pmatrix}
		\pplus\\
		-\pmoins
	\end{pmatrix}
\end{align*}

où on a $R_\theta$ la matrice de rotation d'angle $\theta$ : 

\begin{equation*}
	R_\theta = \begin{pmatrix}
		\cos \theta & -\sin \theta\\
		\sin \theta & \cos \theta
	\end{pmatrix}
\end{equation*}

Donc finalement, 
$\begin{pmatrix}
	P\\
	Z_{c} .u
\end{pmatrix}$ est la rotation d'angle 45° de $\begin{pmatrix}
\pplus\\
-\pmoins
\end{pmatrix}$ dans le sens $\curvearrowleft$, à un facteur près.

D'où, on trouve la courbe $\pplus = G[-\pmoins]$ en faisant tourner la courbe $u=F[P]$ (adimensionnée par l'impédance caractéristique) de 45° dans dans le sens $\curvearrowright$.

\end{document}
