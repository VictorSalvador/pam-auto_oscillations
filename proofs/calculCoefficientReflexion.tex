\documentclass{article}
\usepackage[utf8]{inputenc}
\usepackage[sc]{mathpazo}

\usepackage[T1]{fontenc}
\usepackage{lmodern}
% \renewcommand{\familydefault}{\sfdefault}

%\usepackage[pdftex,
%            pdfauthor={LE Dinh-Viet-Toan},
%           pdftitle={Title},
%            pdfcreator={pdflatex},
%            hidelinks]{hyperref}

\usepackage{ragged2e}
\usepackage[a4paper, right=2.5cm,left=2.5cm,top=2.5cm,bottom=2.5cm]{geometry}

\usepackage{mathtools} % Basic maths
\usepackage{mathabx}
\usepackage{indentfirst} % Indentation at the beginning of a section
\usepackage{textcomp} % Extra characters
\usepackage{todonotes} 
\usepackage{multicol} % Multi columns
\usepackage{cancel} % Draw line to cancel a term

\usepackage[export]{adjustbox}
\usepackage{subcaption}

% Graphics
\usepackage{tikz}
\usetikzlibrary{shapes,arrows,shadings,patterns}
\usepackage{natbib}
\usepackage{graphicx,wrapfig,lipsum}

% Disable number in section
\let\oldsection\section
\renewcommand{\section}[1]{\oldsection*{#1}}
\let\oldsubsection\subsection
\renewcommand{\subsection}[1]{\oldsubsection*{\hspace{1em}#1}}

% Extra commands
\newcommand\norm[1]{\left\lVert#1\right\rVert}
\newcommand{\pplus}{p^{+}}
\newcommand{\pmoins}{p^{-}}
\newcommand{\sqrttwoovtwo}{\frac{\sqrt{2}}{2}}
\newcommand{\powinv}{^{-1}}


% Math operators
\DeclareMathOperator\Imspace{Im}

\providecommand{\dividespace}{\vspace{1.6em}}

\title{Expression de la fonction de réflexion }
\author{}
\date{}



\begin{document}

\maketitle

En reprenant les expressions du champs de pression $p(x, t)$ et du champs de débit $u(x, t)$ en décomposition d'ondes aller et retour \footnote{cf preuve de Viet-Toan}, on a :
\begin{align}
p(x, t) & = p^+(t-x/c)+p^-(t+x/c)\\
u(x,t) & = Z_c^{-1}[p^+(t-x/c)-p^-(t+x/c)].
\end{align}
Où $Z_c=\rho_0c/S$ désigne l'impédance caractéristique d'un tuyau, et $P^{\pm}(\omega, x)=TF[p^{\pm}(t, x)]$. En remarquant que $p^{\pm}(t\pm x/c)=p^{\pm}(t)\ast\delta(t\pm x/c)$, on obtient les expressions de $p$ et $u$ dans le domaine fréquentiel :
\begin{align}
P(\omega, x) & =P^+(\omega)\rm{e}^{-jkx}+P^-(\omega)\rm{e}^{jkx} \label{eq:pfreq}\\ 
U(\omega, x) & = Z_c^{-1}\left[P^+(\omega)^{-jkx}-P^-(\omega)^{jkx}\right]\label{eq:ufreq}.
\end{align}
Avec $k=2\pi fc$ le nombre d'ondes.\par

En évaluant \ref{eq:pfreq} et \ref{eq:ufreq} en $x=l$, et en utilisant la relation $U(\omega, l)=SV(\omega, l)=SZ_r\powinv(\omega)P(\omega, l)$\footnote{J'ai un doute sur cette partie, je ne suis pas sûr que ce soit valable dans le tuyau.} où $Z_r(\omega)$ désigne l'impédance de rayonnement, on obtient

\begin{align}
P(\omega, l) & =P^+(\omega)\rm{e}^{-jkl}+P^-(\omega)\rm{e}^{jkl}\label{pl}\\ 
SZ_r\powinv(\omega)P(\omega, l) & = Z_c^{-1}\left[P^+(\omega)^{-jkl}-P^-(\omega)^{jkl}\right]\label{ul}
\end{align}

En injectant \ref{pl} dans \ref{ul}, on obtient une relation entre $P^+$ et $P^-$ :
\begin{align}
P^+(\omega)\rm{e}^{-jkl}\left[Z_c\powinv-SZ_r\powinv\right]=P^-(\omega)\rm{e}^{jkl}\left[SZ_r\powinv+Z_c\powinv\right]
\end{align}
On en tire une expression de la fonction de réflexion $R(\omega)=TF[r(t)]$:
\begin{align}
R(\omega)=\rm{e}^{2jkl}\frac{P^-(\omega)}{P^+(\omega)}=\frac{Z_c\powinv-SZ_r\powinv}{Z_c\powinv+SZ_r\powinv}
\end{align}\footnote{cf Chaigne\& Kergomard, chap 4 sec 5, eq 4.25}

Le notebook CalcFonctionReflexion illustre la fonction de réflexion dans les domaines temporel et fréquentiel. Il doit y avoir un problème dans le calcul : 
\begin{itemize}
\item $r(t)>0$
\item $R(f)$ est plus passe haut que passe bas
\end{itemize}

\end{document}


