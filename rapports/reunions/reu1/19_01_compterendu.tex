\documentclass[a4paper, 11pt]{article}

\usepackage[french]{babel}
\usepackage[utf8]{inputenc}  
\usepackage[T1]{fontenc}
\usepackage[left=3cm,right=3cm,top=3cm,bottom=3cm]{geometry}
\usepackage{graphicx}
\usepackage{parskip}
\usepackage{titling}

\setlength{\droptitle}{-6em}   % This is your set screw

\title{
	\noindent\rule{\linewidth}{0.4pt}
	\huge{Compte-Rendu Réunion 1\\}
	\medskip
	\Large{PAM --- Auto-Oscillations des Instruments de Musique}
	\noindent\rule{\linewidth}{1pt}
}
\author{Durand, Le, Salvador, Verrier}
\date{19 Janvier 2021}

\begin{document}

\maketitle

Langages à utiliser:
\begin{itemize}
\item Python (général)
\item MaxMsp (temps réel musical)
\end{itemize}

\bigskip

Choix de l'instrument auto-oscillant, sachant que l'on pourra en faire plus (Saxophone, Violon...) mais qu'il vaut mieux bien faire 1 instument plutôt que de s'étaler:
\begin{itemize}
\item Clarinette
\end{itemize}

\bigskip

Étant donné l'instrument, les modèles physiques "excitateur / résonateur" à creuser:
\begin{itemize}
\item modèle avec les décompositions modales (vu en TP)
\item modèle "ligne à retard" du paper McIntyre
\end{itemize}

\bigskip

Cheminement du projet, pour aller de modèle physique à outil prévisible pour un musicien.
\begin{enumerate}
\item Créer une simulation de l'instrument
	\begin{itemize}
	\item Développer un modèle physique (caricatural en premier lieu)
	\item Programmer une simulation en temps continu / différé sur Python
	\end{itemize}
\item Cartographie des régimes
	\begin{itemize}
	\item Choisir l'espace à cartographier (les paramètres du modèle modifiables)
	\item Choisir les descripteurs à utiliser (spectre, loudness...) + caractériser les différents régimes (canard, chaud, brillant...)
	\item Cartographier l'espace des paramètres avec la méthode de mapping choisie (SVM ou autre). On suppose que les paramètres restent constants
	\item (si le temps permet, on pourra essayer d'explorer le cas beaucoup plus difficile où les paramètres changent suivant une trajectoire).
	\end{itemize}
\item Temps réel et contrôleurs
	\begin{itemize}
	\item Traduire le simulateur en temps discret / réel sur Python
	\item Traduire le programme à MaxMsp
	\item Faire une interface tablette / clavier (ex cartographie sur tablette où on peut choisir le régime + clavier donne input midi pour joueur l'instrument simulé)
	\end{itemize}
\end{enumerate}

\bigskip

Objectifs pour la prochaine réunion:
\begin{itemize}
\item approfondir la biblio, lire des segments pertinents du livre de Chaigne et Kergomard
\item bien comprendre les 2 modèles suggérés (décomposition modale + ligne à retard) pour la clarinette
\item première ébauches de simulation sur Python.
\end{itemize}

\end{document}